%@ Subject: Elementary Statistics
\addpoints
\question[30] Una empresa dedicada al mantenimiento de m\'aquinas industriales est\'a encargada de supervisar los equipos utilizados en dos f\'abricas para construir circuitos integrados. Los encargados de cada f\'abrica seleccionaron aleatoriamente un conjunto de 24 m\'aquinas para someterlas a an\'alisis. Luego de realizar los experimentos de rigos, se obtuvo la siguiente informaci\'on:\\$V_1:$ F\'abrica (A,B)\\$V_2:$ Potencia del equipo (B: baja, M: media, A: alta)\\$V_3:$ Temperatura de funcionamiento (Celcius)\\\begin{table}[h!]\centering\begin{tabular}{|c|l|l|l|l|l|l|l|l|l|l|l|l|}\hline M\'aquina & 1 & 2 & 3 & 4 & 5 & 6 & 7 & 8 & 9 & 10 & 11 & 12 \\ \hline $V_1$                    & A & B & A & A & B & B & A & B & B & A  & A  & B  \\ \hline $V_2$                    & A & B & A & M & M & A & B & A & M & A  & B  & A  \\ \hline\end{tabular}\end{table}\begin{table}[h!]\centering\begin{tabular}{|c|l|l|l|l|l|l|l|l|l|l|l|l|}\hline M\'aquina & 13 & 14 & 15 & 16 & 17 & 18 & 19 & 20 & 21 & 22 & 23 & 24 \\ \hline $V_1$                    & A  & B  & B  & A  & A  & B  & A  & B  & B  & A  & B  & A  \\ \hline$V_2$                    & M  & A  & B  & A  & M  & B  & B  & M  & A  & M  & M  & B  \\ \hline\end{tabular}\end{table}\begin{table}[h!]\centering\begin{tabular}{|c|c|c|c|}\hline\multirow{2}{*}{Temperatura de funcionamiento} & \multicolumn{3}{c|}{Potencia del equipo} \\ \cline{2-4}                                                & Baja        & Media        & Alta        \\ \hline$[10-15)$                                      & 2           & 1            & 0           \\ \hline$[15-20)$                                      & 3           & 2            & 1           \\ \hline$[20-25)$                                      & 1           & 2            & 1           \\ \hline$[25-30)$                                      & 1           & 2            & 3           \\ \hline$[30-35)$                                      & 0           & 1            & 4           \\ \hline\end{tabular}\end{table}\noaddpoints\begin{parts}\part[6] Identifique y clasifique cada una de las variables:\part[6] Mediante un gr\'afico apropiado, compare la distribuci\'on de la temperatura de funcionamiento para las potencias baja y alta. Comente.\part[6] Calcule la temperatura promedio para cada nivel de potencia\part[6] Si el an\'alisis se concentra en los equipos de potencia media y alta, ?`Qu\'e porcentaje tiene una temperatura de funcionamiento entre 21 y 33 grados Celcius?\part[6] En un estudio anterior, la temperatura de funcionamiento promedio para equipos de alta potencia alcanz\'o los 26 grados Celcius, con una desviaci\'on est\'andar de 3 grados Celcius. Compare la homogeneidad de la muestra actual con la del estudio mencionado anteriormente.\end{parts}

\begin{solution}
$V_1: \lbrace$ F\'abrica en la que se construyen circuitos integrados $\rbrace$, variable cualitativa en escala nominal.\\$V_2: \lbrace$ Potencia de los equipos utilizados en dos f\'abricas para construir circuitos integrados $\rbrace$, variable cualitativa en escala ordinal.\\$V_3: \lbrace$ Temperatura de funcionamiento en grados Celcius de los equipos utilizados en dos f\'abricas para construir circuitos integrados $\rbrace$, variable cuantitativa continua en escala intervalar.Para datos agrupados se tiene que:$$\overline{x}=\dfrac{\sum_{i=1}^{k}n_i m_i}{n}$$En donde $k$ es la cantidad de clases o intervalos, $n_i$ la frecuencia absoluta de la clase $i-$\'esima y $m_i$ la marca de clase $i-$\'esima. \begin{itemize}\item \textbf{Potencia Baja:} $\overline{x}=\dfrac{12,5*2+17,5*3+\dots+27,5*1}{9}=\dfrac{127,5}{7}=18,2$ $^{\circ}$C\item \textbf{Potencia Media:} $\overline{x}=\dfrac{12,5*1+17,5*2+\dots+32,5*1}{8}=\dfrac{180}{8}=22,5$ $^{\circ}$C\item \textbf{Potencia Alta:} $\overline{x}=\dfrac{17,5*1+22,5*1+\dots+32,5*4}{9}=\dfrac{252,5}{9}=28,1$ $^{\circ}$C \end{itemize} Como el estudio se concentra s\'olo en equipos de potencia Media y Alta, podemos juntar las frecuencias de estas potencias, esto es: \begin{center}\begin{tabular}{|c|c|}\hline Temperatura de Funcionamiento (C) & Frecuencia Absoluta (Potencia Media y Alta) \\ \hline 10-15                             & 1                                           \\ \hline15-20                             & 3                                           \\ \hline 20-25                             & 3                                           \\ \hline 25-30                             & 5                                           \\ \hline 30-35                             & 5                                           \\ \hline \end{tabular}\end{center} Debemos obtener los percentiles asociados a las temperaturas 21 $^{\circ}$C y 33 $^{\circ}$C. Para ello utilizamos:$$P_j=LI+\left(\dfrac{\dfrac{n*j}{100}-N_{i-1}}{n_i}\right)a$$En donde $P_j$ son las temperaturas ($^{\circ}$C), $j$ es el percentil $j-$\'esimo, $n$ el n\'umero total de m\'aquinas de potencia media y alta, $N_{i-1}$ la frecuencia absoluta acumulada hasta la clase percentil anterior y $a$ la amplitud del intervalo. Reemplazando con los datos necesarios, se tiene: \begin{itemize}\item \textbf{Temperatura 21($^{\circ}$C)}:$$21=20+\left(\dfrac{\dfrac{17*j}{100}-4}{3}\right)*5$$Despejando para $j$, se obtiene: $j=27,06 \%$ \item \textbf{Temperatura 33($^{\circ}$C)}:$$33=30+\left(\dfrac{\dfrac{17*i}{100}-12}{5}\right)*5$$Despejando para $i$, se obtiene: $i=88,24 \%$ \\Luego, el porcentaje de equipos de potencia media y alta que tienen temperatura entre 21 $^{\circ}$C y 33 $^{\circ}$C es $(88,24 - 27,06) \% = 61,18 \%$ \end{itemize} Del enunciados sabemos que: $$CV_1= \dfrac{3}{26}=0,12$$En donde $CV_1$ representa el coeficiente de variaci\'on seg\'un un estudio anterior. Necesitamos obtener el coeficiente de varianci\'on de la temperatura de funcionamiento ($^{\circ}$C) de los equipos de alta potencia. Por item c) sabemos que $\overline{x}=28,1^{\circ}$C .Utilizamos:$$S^2=\sum_{i=1}^{n} \dfrac{(x_i-\overline{x})^2}{n-1}$$Y el hecho que los datos est\'an agrupados, por lo que se asume que los datos son la marca de clase del intervalo al que pertenecen. Calculando la varianza:$S^2=27,8 \Rightarrow S=5,3 ^{\circ}$C. Por lo que su coeficiente de varianci\'on es:$$CV_2=\dfrac{5,3}{28,1}=0,19$$Finalmente, se concluye que la homogeneidad de la temperatura de funcionamiento ($^{\circ}$C) para equipos de alta potencia en el estudio anterior es menor a la obtenida en el estudio m\'as reciente, ya que $CV_1 < CV_2$. Apriori, podemos aseverar que la homegeneidad de los datos antiguos es mayor a la actual.
\end{solution}
